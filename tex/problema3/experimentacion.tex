\emph{Nota:} Como vamos a hablar mucho de las distintas podas vamos a denominarlas de la siguiente manera:
\begin{itemize}
\item Primer poda: Terminar cuando lleno el tablero.
\item Segunda poda: Si aunque ponga piezas en los casilleros que me queda revisar, no llego a alcanzar la cantidad de piezas de la solución óptima encontrada a ese momento, corto.
\item Tercer poda: Similar a la segunda poda, pero además revisa hasta $m$ casillas para adelante teniendo en cuenta las restricciones ya existentes en el tablero.
\end{itemize}

BLA BLA BLA

GRAFICOS

BLA BLA BLA

En el gráfico ACA HAY QUE PONER EL NUMERO CORRECTO, se puede observar claramente que la tercer poda logra podar mas casos, pero no logra compensar el costo adicional de $O(m)$. Dicha poda podría servir en otro problema en el que cual cada llamada recursiva tenga un mayor costo.