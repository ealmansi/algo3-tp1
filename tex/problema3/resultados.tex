Para concluir, nos parece importante recalcar el impacto de las podas en este problema. Es simple realizar un algoritmo que recorra todas las soluciones y así encuentre la óptima pero esto incurriría en un algoritmo muy tosco y lento. Las podas nos ayudan a determinar si seguir por una rama del árbol de soluciones del problema nos llevará a una solución que valga la pena (con que valga la pena nos referimos a que la solución a la que nos lleve sea mejor que la que ya tengamos) y de no ser así podemos descartar dicha rama ahorrándonos varios casos.

Habiendo dicho eso, por mas que supongamos que no hay problemas en la implementación de las podas (es decir, no haya \emph{bugs}), puede ser que utilizar una cierta poda puede no llegar a ser beneficioso. Puede llegar a ocurrir que, si bien una poda nos ayude a descartar muchas ramas del árbol de soluciones posibles, dicha poda sea muy ineficiente en términos de complejidad temporal y por consiguiente, el algoritmo final termine siendo más lento que una solución mas \emph{naïf}.

Para terminar, vale la pena aclarar que utilizar nuestras podas es efectivamente mejor que no utilizar poda alguna. A su vez, dentro del contexto del problema 3, utilizar la segunda o tercer poda no tuvo una ventaja considerable por sobre la otra y va a depender del programador elegir una u otra.