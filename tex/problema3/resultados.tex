Para concluir, nos parece importante recalcar el impacto de las podas en este problema. Es simple realizar un algoritmo que recorra todas las soluciones y así encuentre la óptima pero esto incurriría en un algoritmo muy tosco y lento. Las podas nos ayudan a determinar si seguir por una rama del árbol de soluciones del problema nos llevará a una solución que valga la pena (con que valga la pena nos referimos a que la solución a la que nos lleve sea mejor que la que ya tengamos) y de no ser así podemos descartar dicha rama ahorrándonos varios casos.

Habiendo dicho eso, creemos que es importante tener en cuenta que también importa cómo se implementan dichas podas. Puede llegar a ocurrir que, si bien una poda nos ayude a descartar muchas ramas del árbol de soluciones posibles, dicha poda sea muy ineficiente en términos de complejidad temporal y por consiguiente, el algoritmo final termine siendo más lento que la solución \emph{naïf}.