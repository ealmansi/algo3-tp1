Para concluir, nos parece importante recalcar el impacto de las podas en este problema. Es simple realizar un algoritmo que recorra todas las soluciones y así encuentre la óptima pero esto incurriría en un algoritmo muy tosco y lento. Las podas nos ayudan a descartar casos que no valen la pena tener en cuenta ya que por esa rama del árbol no vamos a encontrar nada mejor.

Habiendo dicho eso, creemos que es importante tener en cuenta que también importa cómo se implementan dichas podas. Puede llegar a ocurrir que, si bien una poda pode muchas ramas del árbol de soluciones posibles, dicha poda sea muy ineficiente y por consiguiente, el algoritmo final sea peor que la solución \emph{naïf}.