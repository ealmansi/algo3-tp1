Para resolver el problema implementamos el siguiente algortimo:

\begin{pseudo}
\State Tipo de dato Pieza es Tupla $\langle$ id : entero, inferior : entero, superior : entero, izquierda : entero, derecha : entero $\rangle$
\State Tipo de dato Tablero es Tupla $\langle$ cantFichas : entero, casillas : Matriz$\langle$Pieza$\rangle$ $\rangle$
    \Procedure{Rompecolores}{$n,m,c,\langle p_1, \ldots, p_{n*m} \rangle$}
        \State SolucionParcial, SolucionOptima$\leftarrow$ nuevo Tablero
        \State ResolverBacktracking(SolucionOptima,SolucionParcial,0,0,...) 
        \State \textbf{return} SolucionOptima
    \EndProcedure
\end{pseudo}

\begin{pseudo}

    \Procedure{ResolverBacktracking}{SolucionOptima,SolucionParcial,i,j,...}
        \For{Cada Ficha}
	  \If{EstaDisponible $\wedge$ esCompatible}
	    \State Agrego ficha a SolucionParcial
	    \If{SolucionParcial mejor que SolucionOptima}
	      \State SolucionOptima = SolucionParcial
	    \EndIf
	    \If{!Podo}
	      \State ResolverBacktracking en siguiente posicion
	    \EndIf
	    \State Remuevo ficha 
	  \EndIf
	\EndFor
        \State Agrego ficha blanca a SolucionParcial
        \State ResolverBacktracking en siguiente posicion
    \EndProcedure
\end{pseudo}

Donde: 

$\bullet$ EstaDisponible devuelve si la ficha no fue usada todavia (verdadero), o si ya fue agregada al tablero (falso).

$\bullet$ esCompatible devuelve verdadero si al agregar la ficha en la posicion en la cual se esta trabajando en el tablero SolucionParcial, no se genera ninguna contradiccion.

$\bullet$ Podo devuelve verdadero si a travez de alguna de nuetras podas podemos asegurar que en esa rama no se encuentra ninguna solucion mejor a SolucionOptima.

Implementamos tres posibles funciones Podo(), la primera es la mas simple que solo devuelve falso cuando se llega al final del tablero. Si bien no poda, la utilizamos como corte en el caso en que encuentre una solución donde las fichas llenan todo el tablero, por lo que seguir recorriendo es inútil.

La segunda opción se fija si la cantidad de fichas que puse en el tablero más la cantidad de lugares del tablero que todavía no recorrí, es menor que la cantidad de fichas de SolucionOptima. Si es menor sabemos que no hay ninguna forma que por esa rama se llegue a una solución mejor que SolucionOptima y por esto, la descartamos.

La tercer poda que realizamos incluye a la anterior pero realiza algo extra: En las posiciones que todavía no recorrimos pero ya sabemos que tienen una restricción (como por ejemplo, que necesitan el color 1 en superior), queremos saber cuantas fichas tenemos disponible para esos lugares. Para esto contamos para cada color, cuantos de estos lugares necesitan una ficha con ese color en la posición superior. Luego recorremos las fichas disponibles y restamos 1 al nuemero calculado anteriormente para ese color.

Si sumamos todos los contadores de color que no nos quedaron negativos, conseguimos el numero de estos casilleros a los que no le vamos a poder agregar fichas. Luego, el numero de fichas posibles calculado en la opción 2 le restamos lo que acabamos de calcular obteniendo una cota mas ajustada de la cantidad de fichas que puede llegar a tener esta rama del árbol.

Estas podas nos ayudan a descartar casos para no tener que observarlos. En la etapa de experimentación vamos a evaluar si el uso de estas podas es favorable o no. Esto se debe a que por más que se pode una rama (o algunas ramas), la poda utilizada puede llegar a ser más costosa que recorrer dicha rama (o ramas).