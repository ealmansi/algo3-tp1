Hallamos que el orden óptimo es de menor a mayor según el coeficiente $\frac{Tiempo_i}{Perdida_i}$, lo cual está demostrado en su sección correspondiente. Siguiendo este orden, podemos asegurar que una solución óptima cumple con el siguiente invariante:

\begin{center}

$(\forall i \in [1..\#{piezas})) \frac{Tiempo_i}{Perdida_i} < \frac{Tiempo_{i+1}}{Perdida_{i+1}}$
 
\end{center}

Lo cual es equivalente a:

\begin{center}

$(\forall i \in [1..\#{piezas})) Tiempo_i * Perdida_{i+1} < Tiempo_{i+1} * Perdida_i$
 
\end{center}

Este último es el que usamos debido a que al usar enteros, preferimos usar multiplicación antes que división.

Nuestro algoritmo crea un vector, y coloca una a una las piezas una atrás de otra. Luego, ordena según el orden previamente enunciado. Finalmente, calcula la pérdida total. El pseudocódigo de nuestro algoritmo es el siguiente:

\renewcommand{\algorithmiccomment}[1]{\hskip2em$//$ #1}

\begin{pseudo}
\State Tipo de dato Pieza es Tupla $\langle$ id : entero, pérdida : entero, tiempo : entero $\rangle$
    \Procedure{La joya del Río de la Plata}{$\langle p_1, \ldots, p_n \rangle, L$}
        \State $V \leftarrow$ nuevo vector de Pieza %\Ode{1}
        \State Cargo\_Piezas(V) %\Ode{n}
        \State Sort(V) %\Ode{n*log(n)}
        \Comment Este sort se hace de menor a mayor según el coeficiente anteriormente dicho
        \State \textbf{return} Calcular\_perdida(V) %\Ode{n}
    \EndProcedure
\end{pseudo}
