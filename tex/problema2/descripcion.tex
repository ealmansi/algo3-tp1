Este problema se trata de un joyero que debe fabricar un conjunto de piezas para luego venderlas. 

La problemática radica en que cada pieza \emph{i} de este conjunto tiene una cantidad de días que requiere para su fabricación  (\emph{$t_i$}), y además cada una pierde una fracción de su valor (\emph{$p_i$})por cada día que pasa.

Lo que este problema nos pide hacer un algoritmo que determine un orden para la fabricación de estas piezas que minimice las perdidas, y además mostrar cuál es dicha pérdida. La complejidad del algoritmo utilizado debe ser $O(n^2)$.

A continuación vamos a dar un ejemplo del problema planteado junto con su solución. Supongamos que tenemos las siguientes piezas:

\begin{tabular}{|c|c|c|}
 \hline
 \textbf{Pieza} & \textbf{Pérdida} & \textbf{Tiempo} \\
 \hline
 1 & 3 & 1 \\
 
 2 & 2 & 1 \\
 
 3 & 1 & 1 \\
 \hline
\end{tabular}

En este ejemplo la solución es la siguiente secuencia:

\textbf{Solución} = [Pieza 1, Pieza 2, Pieza 3]

La pérdida total para esta solución es: $3*1 + 2*2 + 1*3 = 10$

En cambio si eligiéramos como solución otra secuencia, como por ejemplo [Pieza 2, Pieza 1, Pieza 3], la pérdida total sería de: $2*1 + 3*2 + 1*3 = 11$. Si eligieramos [Pieza 3, Pieza 2, Pieza 1], la pérdida total sería de: $1*1 + 2*2 + 3*3 = 14$.