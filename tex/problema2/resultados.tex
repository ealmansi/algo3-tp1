Para concluir, nos parece importante destacar que la complejidad temporal de nuestra solución es dominada por el algoritmo de ordenamiento. Esto implica que no va a ser posible mejorar la cota de complejidad teórica de nuestra solución debido a que los algoritmos de sorting son, a lo sumo, $O(n*log(n))$. Sin embargo, cabe destacar que en el caso en que los datos están ordenados al momento de leer, podemos mejorar la complejidad temporal de $O(n * log n)$ a $O(n)$.

Por otro lado, sin contar la etapa de ordenamiento, realizamos a lo sumo $O(n)$ operaciones, ya que recorremos el vector por lo menos dos veces: una para guardar los datos y otra para saber cuál es la pérdida total. Por esto, podemos asegurar que por más que los datos vengan ordenados no vamos a poder mejorar la complejidad temporal de $O(n)$.