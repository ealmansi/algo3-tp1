q.v.q. Sea S un conjunto de elementos  ${s_1,...,s_n}$ y T un permutación de los elementos de S / $\forall 1 \leq i  < n$ $\frac{tiempo(T[i])}{perdida(T[i])} \leq \frac{tiempo(T[i+1])}{perdida(T[i+1])} $ optimiza la función C(T) 
\\
Siendo C(T) = $\sum_{i=1}^{n} tiempo(T[i]) \sum_{j=i}^{n}perdida(T[j]) $
\\
Caso base $\|$ T$\|$ = 1 :
\\
Solo existe una permutacion de T, por lo cual claramente es la optima.
\\
\\
Paso inductivo:
$\forall n > 1$P(n-1) $\Rightarrow$ P(n)
\\
Tomo un permutacion optima T / T = $(t_1,...,t_n)$ y construyo T' = $(t'_1,...,t'_n) / t'_1 = t_1 y (t'_2,...,t'_n) es una permutacion de (t_2,...,t_n) / \forall 2 \leq i  < n \frac{tiempo(t'_i)}{perdida(t'_i)} \leq \frac{tiempo(t'_{i+1})}{perdida(t'_{i+1})} $
\\
Primero q.v.q. T' es optima para esto calculo C(T) y C(T')
\\
C(T) = $tiempo(t_1) \sum_{j=1}^{n}perdida(t_j) + C(T[2,n]) $
C(T') = $tiempo(t_1) \sum_{j=1}^{n}perdida(t'_j) + C(T'[2,n]) $
\\
Como T es optimo, se que C(T') $\leq$ C(T)
\\
Tambien se por H.I. que T'[2,n] es optima, por lo cual 
\\
C(T'[2,n]) $\leq$ C(T'[2,n])
$tiempo(t_1) \sum_{j=1}^{n}perdida(t_j)$ + C(T'[2,n]) $\leq$  $tiempo(t_1) \sum_{j=1}^{n}perdida(t_j)$ + C(T'[2,n])
$\sum_{j=1}^{n}perdida(t_j) = \sum_{j=1}^{n}perdida(t'_j) ya que T' es una permutacion de T, Entonces$
$tiempo(t_1) \sum_{j=1}^{n}perdida(t'_j)$ + C(T'[2,n]) $\leq$  $tiempo(t_1) \sum_{j=1}^{n}perdida(t_j)$ + C(T'[2,n])
C(T') $\leq$ C(T)

C(T') $\leq$ C(T) $\wedge$ C(T') $\leq$ C(T) $\leftarrow$ C(T) = C(T'), osea T' es optimo.

Por ultimo queremos ver que T' cumple con la condicion de P(n), sabemos que T'[2,n] estan ordenados segun $\frac{tiempo(t'_i)}{perdida(t'_i)}$. Nos falta ver que T' completa lo esta, para esto solo hace falta ver $t'_1$ esta ordenado, que es lo mismo que decir que $\frac{tiempo(t'_1)}{perdida(t'_1)}$ $\leq \frac{tiempo(t'_2)}{perdida(t'_2)}$

Para esto tomamos T'' = $(t'_2,t'_1,t'_3,...,t'_n)$
C(T') = $tiempo(t'_1) \sum_{j=1}^{n}perdida(t'_j) + tiempo(t'_2) \sum_{j=2}^{n}perdida(t'_j) + C(T'[3,n]) $
C(T'') = $tiempo(t'_2) \sum_{j=1}^{n}perdida(t'_j) + tiempo(t'_1) \sum_{j=1,j\neq 2}^{n}perdida(t'_j) + C(T''[3,n]) $
C(T') $\leq$ C(T'') por ser T' optimo
$tiempo(t'_1) \sum_{j=1}^{n}perdida(t'_j) + tiempo(t'_2) \sum_{j=2}^{n}perdida(t'_j) + C(T'[3,n]) \leq tiempo(t'_2) \sum_{j=1}^{n}perdida(t'_j) + tiempo(t'_1) \sum_{j=1,j\neq 2}^{n}perdida(t'_j) + C(T''[3,n])$
$tiempo(t'_1) \sum_{j=1}^{n}perdida(t'_j) + tiempo(t'_2) \sum_{j=2}^{n}perdida(t'_j)  \leq tiempo(t'_2) \sum_{j=1}^{n}perdida(t'_j) + tiempo(t'_1) \sum_{j=1,j\neq 2}^{n}perdida(t'_j) $ por ser T'[3,n] = T''[3,n]
$tiempo(t'_1) \sum_{j=1,j\neq 2}^{n}perdida(t'_j) + tiempo(t'_1) * perdida(t'_2) + tiempo(t'_2) \sum_{j=2}^{n}perdida(t'_j) \leq tiempo(t'_1) \sum_{j=1,j\neq 2}^{n}perdida(t'_j) + tiempo(t'_2) \sum_{j=2}^{n}perdida(t'_j)+ tiempo(t'_2) * perdida(t'_1)$  
$tiempo(t'_1) * perdida(t'_2) \leq tiempo(t'_2) * perdida(t'_1) $ cancelando
$\frac{tiempo(t'_1)}{perdida(t'_1)} \leq \frac{tiempo(t'_2}{perdida(t'_2)}$

Con lo que finaliza la demostracion



