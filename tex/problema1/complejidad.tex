NO SE DONDE PONER LO DEL SORT XQ LO USAMOS EN LOS PROB 1 Y 2.

En los algoritmos 1 y 2 usamos el sort de la stl. Para lograr calcular las complejidades de ambos, necesitamos saber la complejidad del sort. Buscando en ``AGREGAR DONDE ENCONTRAMOS ESTA INFO'', encontramos que su complejidad es O(n.log n ) comparaciones. Como con solo esta informacion no podiamos asegurar que tenga complejidad O(n.log n) operaciones, por eso buscamos que hacia el sort de stl. Encontramos que para casos chicos hacia InsertionSort (ERA ESTE?), y en casos mas grandes IntroSort. IntroSort intenta ordenar usando QuickSort, si no lo resuelve en n.log n pasos, usa HeapSort para garantizar O(n.log n) comparaciones. Viendo el código del algortimo llegamos a que ademas de hacer O(n.log n) comparaciones tambien hace a lo sumo O(n.log n) swaps.
Como en ambos casos donde usamos el algoritmo sort de la stl, nuestros parametros son vector<int> sabemos que la comparaciones y swaps son O(1). Por lo cual podemos garantizar que sort tiene una complejidad O(n log n) operaciones