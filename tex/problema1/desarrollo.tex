El algoritmo utilizado para resolver el problema opera de la siguiente forma: se consideran como posibles candidatos para $d$ únicamente los valores $d_1\;d_2\;...\;d_n$, y para cada $i$ de $1$ a $n$ se computa la cantidad de elementos contenidos en $[d_i, d_i + D)$. En cada paso, se mantiene un registro del valor de $d$ para el cual esta cantidad es mayor, y finalmente se emite como solución dicho valor y la cantidad de números contenidos en su respectivo intervalo.

Una posible manera para llevar a cabo este procedimiento consiste en tomar cada valor $d_i$ y, para cada valor $1 \leq j \leq n$, evaluar la condición de pertenencia $d_i \leq d_j < d_i + D$. Esto conlleva una cantidad de operaciones cuadrática en $n$, por lo que se optó por el siguiente algoritmo (expresado en pseudocódigo).

\begin{pseudo}
    \Procedure{Camiones-Sospechosos}{$D, n, \langle d_1, \ldots, d_n \rangle$}
        \State $ordenar(\langle d_1, \ldots, d_n \rangle)$ \Ode{n\;log(n)}
        \State $d \leftarrow 0, \#camiones \leftarrow 0$ \Ode{1}
        \State $i \leftarrow 1$, $j \leftarrow 1$ \Ode{1}
        \While{$i \leq n$} \Ode{1}
            \While{$(j \leq n) \land (d_i \leq d_j < d_i + D)$} \Ode{1}
                \State $j \leftarrow j + 1$ \Ode{1}
            \EndWhile
            \If{$\#camiones < j - i$} \Ode{1}
                \State $\#camiones \leftarrow j - i$ \Ode{1}
                \State $d \leftarrow i$ \Ode{1}
            \EndIf
            \State $i \leftarrow i + 1$ \Ode{1}
        \EndWhile
    \EndProcedure
\end{pseudo}

