Para concluir, nos parece importante destacar que la complejidad temporal de nuestra solución es dominada por la etapa de ordenamiento, lo cual provee una cota inferior a la complejidad temporal de nuestra solución, debido a que los algoritmos de ordenamiento por comparaciones son $\Omega(n * log(n))$. Sin embargo, en el caso en que los datos de entrada se encuentren ordenados o dentro de un rango acotado\footnote{Considerar casos donde la entrada solo admite días dentro de un rango es razonable dado el dominio del problema.}, la instancia se puede resolver en tiempo $O(n)$.

%  esto lo comento porque la etapa de lectura de datos no la incluímos en el análisis teórico, y no estoy seguro de que es imposible resolver el problema en tiempo mejor que O(n) para entradas ordenadas.
% Por otro lado, sin contar la etapa de ordenamiento, realizamos a lo sumo $O(n)$ operaciones, ya que podríamos tener que recorrer el vector por lo menos dos veces: una para guardar los datos y otra para saber cuál es el día óptimo para colocar al inspector. Por esto, podemos asegurar que por más que los datos vengan ordenados no vamos a poder mejorar la complejidad temporal de $O(n)$.