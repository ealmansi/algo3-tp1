Para demostrar la correctitud del algoritmo, vamos a demostrar 2 propiedades:

Primero queremos ver que siempre existe una solucion optima que empieza en algunos de los dias dados(Propiedad 1).

$(\forall \ n,\ D\ , S = \{d_1,..,d_n\} ) \ \exists \ 0 < i \leq n \ / \ [d_i,d_i + D )$ es optimo 

Tomamos cualquier solución óptima T que empieza en el día d'.
Si $(\exists i: 0 < i\leq n ) \ d' = d_i$ para algún 0 $<$ i $\leq$ n, encontramos lo que buscabamos.

Si $(\forall i: 0<i \leq n) \  d' \neq d_i$, dividimos en 2 subcasos

Si $(!\exists i: 0<i\leq n) \ d' < d_i$ entonces la solución T no tiene ningún elemento, ya que no hay ningún elemento $d_i \in S$ tal que $d' \leq d_i < d' + D$. Por lo tanto si tomo una solución T' que contenga a un día de S como por ejemplo una que empiece en $d_i$ esta solución sería mejor que la solución T lo cual es absurdo ya que habíamos dicho que T era óptima.  

El otro subcaso es en el que $(\exists i: 0<i\leq n) \ d' < d_i$. En ese caso tomamos el mínimo i / d' $< d_i$
Ahora construimos una nueva solución desde $d_i$, veamos que todo elemento de S perteneciente a [d',d'+D) también pertenece a $[d_i,d_i +D)$

sea $d_j \in [d',d'+D) \Rightarrow d' \leq d_j < d'+D $

q.v.q.$ d_j \in  [d',d'+D) \Rightarrow d_i \leq d_j < d_i+D $

%$( d' \leq d_j < d'+D \Rightarrow d_i \leq d_j < d_i+D)  \Leftrightarrow$    
%$( d' \leq d_j \Rightarrow d_i \leq d_j)$  como sabemos que $d' < d_i$
%este es valido ya que sabemos que $d_i$ es el minimo elemento de S mayor que d' $\Rightarrow$ 
%$( !\exists k: 0 < k \leq n) d' < d_j < d_i$

Esto es lo mismo que probar que $(d' \leq d_j < d'+D) \Rightarrow (d_i \leq d_j < d_i+D) $, que a su vez es lo mismo que probar que $(d' \leq d_j \Rightarrow d_i \leq d_j) \wedge (d_j < d'+D \Rightarrow d_j < d_i+D)$

Primero vamos a probar que $d' \leq d_j \Rightarrow d_i \leq d_j$:

Como sabemos que $d' < d_i$ y sabemos que $d_i$ es el mínimo elemento de S mayor que $d'$ entonces si $d' \leq d_j$ tiene que pasar que $d' < d_i \leq d_j$, de lo contrario $d_j$ no pertenecería al conjunto S.

Ahora vamos a probar que $d_j < d'+D \Rightarrow d_j < d_i+D$:

Como $d' < d_i$ entonces $d'+D < d_i+D$ por lo tanto $d_j<d'+D<d_i+D \Rightarrow d_j<d_i+D$.

Ahora queremos demostrar la propiedad 2 : $((\forall k: i \leq k < j)\;d_k \in [d_i, d_i + D)) \Rightarrow ((\forall k: i + 1 \leq k < j)\;d_k \in [d_{i+1}, d_{i+1} + D))$

Vamos a demostrar esto por el absurdo. Tomando como verdadero el antecedente y suponiendo falso el consecuente.

Para eso suponemos que $(\exists k': i+1\leq k' < j) \;d_{k'} \notin [d_{i+1}, d_{i+1} + D) $

Primero vamos a enumerar varias propiedades que sabemos por el antecedente y porque los elementos $d_i$ están ordenados

A : $d_i < d_{i+1}$

B : $d_{i+1} \leq d_{k'}$ ya que $i+1 \leq k'$

C : $d_{k'} < d_i +D$ ya que $d_{k'} \in [d_i,d_i +D)$, por el antecedente

q.v.q $d_{k'} \notin [d_{i+1}, d_{i+1} + D)$ es absurdo.

Equivalentemente: $d_{k'} < d_{i+1} \vee d_{k'} \geq d_{i+1} + D $ 

$d_{k'} < d_{i+1}$ es falso por B.

$d_{k'} \geq d_{i+1} + D  > d_i + D$ por A,

$d_{k'} > d_i + D$ lo cual es falso por C.

Al ser falsas ambas partes del $\vee$, la afirmación es falsa, lo cual nos lleva al absurdo que queríamos encontrar.

