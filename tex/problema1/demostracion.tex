Para demostrar la correctitud del algoritmo propuesto, es necesario probar esencialmente dos afirmaciones mencionadas en la sección \ref{problema1-desarrollo}. En primer lugar, debe justificarse la decisión de cuáles candidatos se eligen como posibles fechas de inicio del período de contratación; o, formalmente, se debe demostrar la Propiedad \ref{propiedad-candidatos}. Luego, es importante argumentar por qué el ciclo del algoritmo computa efectivamente la cantidad de elementos de la lista contenidos en el intervalo asociado a cada candidato.

\subsubsection{Demostración de la Propiedad \ref{propiedad-candidatos}}

Dada una instancia $<D,\;n,\;l = [d_1, d_2, ..., d_n]>$, se quiere probar que siempre existe un intervalo óptimo $[d, d + D)$ (en el sentido de que comprende la mayor cantidad posible de elementos de la lista $l$) cuyo límite inferior es $d \in \{d_1,d_2,...,d_n\}$.

Como cualquier valor $d$ natural genera un intervalo $[d, d + D)$ válido, el conjunto de posibles soluciones es no vacío. Como además cualquier solución posible contiene entre 0 y $n$ elementos de la lista, necesariamente existe solución óptima. Sea $d'$ tal que $[d', d' + D)$ es un intervalo óptimo.

\begin{itemize}
  \item Si $(\nexists\;i : 1 \leq i \leq n)\;d' \leq d_i$, entonces el intervalo $[d', d' + D)$ no contiene ningún elemento de la lista. Esto es absurdo porque cualquier intervalo $[d_i, d_i + D)$ contiene al menos un elemento, y sería mejor que el óptimo.
  \item De lo contrario, se puede definir $d^*$ como el menor elemento de la lista que es mayor o igual a $d'$; es decir: $d^* = min\;\{\;d_i : 1 \leq i \leq n \land d' \leq d_i\;\}$.

  Luego, $\forall\;i : 1 \leq i \leq n$:
  
  \begin{itemize}
    \item Como $d^*$ es el menor elemento que cumple la propiedad de ser mayor o igual que $d'$:
    $$d' \leq d_i \rightarrow d^*\leq d_i$$
    \item Como $d' \leq d^*$ por definición, se desprende que $d' + D \leq d^* + D$, y por lo tanto:
    $$(d_i < d' + D) \rightarrow (d_i < d^* + D)$$
  \end{itemize}
  Entonces, vale que todo elemento contenido en el intervalo óptimo también está contenido en el intervalo $[d^*, d^* + D)$. Como $d^* \in \{d_1,d_2,...,d_n\}$, queda demostrada la propiedad.
\end{itemize}

\subsubsection{Correctitud del ciclo}

En la inicialización previa al ciclo, se configura $(i, j) \leftarrow (1, 1)$ y por lo tanto, vale trivialmente el invariante:

$$(\forall k: i \leq k < j)\;d_k \in [d_i, d_i + D)$$

El cuerpo del ciclo se ejecuta únicamente para el primer elemento de la lista, y para todos aquellos que no sean iguales al inmediatamente anterior. Como la lista está ordenada, esto es equivalente a evaluar únicamente la primera aparición de cada valor dentro de la lista, ignorando sus posibles repeticiones posteriores.

El ciclo interno tiene la única función de incrementar el valor de $j$. La guarda del mismo es:

$$(j \leq n) \land (d_i \leq d_j < d_i + D) \equiv (j \leq n) \land d_j \in [d_i, d_i + D)$$

Es decir, $j$ jamás se incrementa si el $j$-ésimo elemento no pertenece al intervalo $[d_i, d_i + D)$. Esto mantiene el invariante, evitando que algún elemento con índice menor a $j$ no pertenezca al intervalo del candidato en evaluación. Como además $j$ se incrementa tantas veces como sea posible sin romper el invariante, esto garantiza que al terminar el ciclo interno la variable tendrá su valor máximo (cuando $j = n + 1$ ó $d_j \notin [d_i, d_i + D)$).

Luego, $d_i$ es el primer elemento de la lista contenido en $[d_i, d_i + D)$ por no tener predecesor o por ser mayor que el mismo. Además, $d_{j-1}$ es el último elemento de la lista contenido en $[d_i, d_i + D)$ por no tener sucesor, o porque lo garantiza el invariante en caso contrario. De ello se desprende que existen $(j - 1) - i + 1 = j - i$ elementos de la lista contenidos en el intervalo, por lo cual es el valor que se utiliza para actualizar el estado de la solucion parcial.

Por último, es necesario probar que el invariante se mantiene al incrementar $i$ al final del ciclo. Esto es equivalente a demostrar la propiedad \ref{propiedad-maximo-j}.

\subsubsection{Demostración de la Propiedad \ref{propiedad-maximo-j}}

Si el invariante de ciclo es válido para el par $(i,j)$, entonces también lo es para $(i + 1,j)$. Formalmente:

$$((\forall k: i \leq k < j)\;d_k \in [d_i, d_i + D)) \Rightarrow ((\forall k: i + 1 \leq k < j)\;d_k \in [d_{i+1}, d_{i+1} + D))$$

De la verdad del antecedente, y dado que los elementos $d_i$ están ordenados al comenzar el ciclo, se deducen las siguientes propiedades para todo $k'$ tal que $i + 1 \leq k'< j$:

\begin{enumerate}
  \item $d_{i+1} \leq d_{k'}$               \hfill        por estar ordenados, ya que $i+1 \leq k'$
  \item $d_i \leq d_{i+1}$                  \hfill        por estar ordenados
  \item $d_{i} + D \leq d_{i + 1} +D$       \hfill        se deduce de 2
  \item $d_{k'} < d_i +D$                   \hfill        por el antecedente, ya que $d_{k'} \in [d_i,d_i +D)$
  \item $d_{k'} < d_{i+1} +D$               \hfill        se deduce de 3 y 4
\end{enumerate}

En base a 1 y 5, vale que $d_{k'} \in [d_{i+1}, d_{i+1} + D)$, y queda demostrada la propiedad.