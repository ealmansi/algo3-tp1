Para demostrar la correctitud del algoritmo, vamos a demostrar 2 propiedades:

Primero queremos ver que siempre existe una solucion optima que empieza en algunos de los dias dados.

$(\forall \ n,\ D\ , S = \{d_1,..,d_n\} ) \ \exists \ 0 < i \leq n \ / \ [d_i,d_i + D )$ es optimo 

Tomo cualquier solucion optima T que empieza en el dia d'.
Si $(\exists i: 0 < i\leq n ) \ d' = d_i$ para algun 0 $<$ i $\leq$ n, encontre lo que buscaba.

Si $(\forall i: 0<i \leq n) \  d' \neq d_i$, divido en 2 subcasos

Si $(\exists i: 0<i\leq n) \ d' < d_i$, tomo el minimo i / d' $< d_i$
Ahora construyo una nueva solucion desde $d_i$, veamos que todo elemento de S perteneciente a [d',d'+D) tambien pertenece a $[d_i,d_i +D)$

sea $d_j \in [d',d'+D) \Rightarrow d' \leq d_j < d'+D $

q.v.q.$ d_j \in  [d',d'+D) \Rightarrow d_i \leq d_j < d_i+D $

$ d' \leq d_j < d'+D \Rightarrow d_i \leq d_j < d_i+D  \Leftrightarrow$    
$ d' \leq d_j \Rightarrow d_i \leq d_j$  como sabemos que $d' < d_i$
este es valido ya que sabemos que $d_i$ es el minimo elemento de S mayor que d' $\Rightarrow$ 
$( !\exists k: 0 < k \leq n) d' < d_j < d_i$

Ahora queremos demostrar la propiedad 2 : $((\forall k: i \leq k < j)\;d_k \in [d_i, d_i + D)) \Rightarrow ((\forall k: i + 1 \leq k < j)\;d_k \in [d_{i+1}, d_{i+1} + D))$