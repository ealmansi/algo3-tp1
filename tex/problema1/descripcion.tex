Bajo la sospecha de que una determinada empresa de transporte trafica sustancias ilegales en sus camiones, se ha contratado a un experto para inspeccionar los vehículos durante su paso por un puesto de control. El período de contratación del experto tiene una duración de $D$ días consecutivos, con fecha de inicio a determinar. Adicionalmente, se cuenta con un listado de $n$ entradas, indicando los días $d_1, d_2, ..., d_n$ en que los camiones de la empresa pasarán por el puesto de control.

Se desea conocer la máxima cantidad $c$ de camiones que el experto puede llegar a inspeccionar durante su período de contratación, y un día $d$ para tomar como fecha de inicio de forma tal que efectivamente logre inspeccionar dicha cantidad de camiones.

Formalmente, dado $D$ un numero natural y una lista no vacía de $n$ números naturales $d_1, d_2, ..., d_n$ (no necesariamente distintos), se desea hallar un valor $d$ natural de forma tal que el intervalo $[d, d + D)$ contenga a la máxima cantidad posible de elementos de la lista. Además, se desea conocer la cantidad total $c$ de elementos contenidos en el intervalo.

Cada instancia del problema, así como su solución, se codifica como una lista de naturales separados por espacios, representando los siguientes valores:

\begin{tabular}{ll}
Entrada: &$D\;n\;d_1\;d_2\;...\;d_n$ \\
Salida:  &$d\;c$
\end{tabular}

\subsubsection{Ejemplos y observaciones}

A partir de cualquier $d$ natural se puede tomar un intervalo válido, por lo cual el conjunto de posibles soluciones es no vacío. Como además la cantidad de elementos que puede contener cualquier intervalo está acotada entre $0$ y $n$, siempre existe al menos una solución óptima. Sin embargo, esta no tiene por qué ser única. Se muestra un ejemplo con múltiples soluciones óptimas:

\begin{tabular}{ll}
Entrada: &$3\;2\;7\;6$            \\
Salida:  &$5\;2$ \hspace{10mm} ó  \\
Salida:  &$6\;2$ \hspace{10mm}
\end{tabular}

En el ejemplo anterior se observa que la lista no necesariamente se encuentra ordenada. Adicionalmente, esta puede contener repetidos que deben ser contados con su debida multiplicidad al computar la solución.

\begin{tabular}{ll}
Entrada: &$1\;4\;23\;23\;23\;23$            \\
Salida:  &$23\;4$ \hspace{10mm}
\end{tabular}

% Nulla facilisi. Pellentesque a velit nisi. Quisque nunc tellus, dictum sed nulla quis, molestie elementum nulla. Pellentesque in metus vulputate urna feugiat molestie. Proin id rhoncus lorem. Sed fringilla, enim tincidunt hendrerit condimentum, quam risus euismod mauris, a rutrum urna mi non lorem. Ut eget mi urna. Nulla ac turpis sit amet nibh pulvinar scelerisque.

% Curabitur sagittis semper augue. Vestibulum ante ipsum primis in faucibus orci luctus et ultrices posuere cubilia Curae; In sed hendrerit lectus. Pellentesque hendrerit sem sit amet tellus lobortis cursus. Fusce vel pellentesque diam, non pulvinar erat. Ut gravida diam nec ligula molestie, non dapibus tortor sollicitudin. Ut condimentum tempor arcu eget pretium. Fusce accumsan neque ac lobortis cursus. Duis nec nibh vitae ante lobortis dignissim id iaculis augue. Curabitur in nisl massa. Curabitur eget pretium lorem.

% Vivamus adipiscing diam quis dolor tempor consectetur. Sed porttitor ante ipsum, ac sagittis urna eleifend in. Nam rutrum ac ligula nec pharetra. Nullam a urna nec est molestie aliquam nec a ligula. Nullam non massa posuere erat aliquam condimentum vel sed nisi. Ut fringilla viverra scelerisque. Phasellus mattis tristique sapien, nec pharetra velit porttitor ac. Praesent fringilla tempor accumsan. Aenean lorem sapien, luctus id purus a, egestas dapibus eros.
