Bajo la sospecha de que una determinada empresa de transporte trafica sustancias ilegales en sus camiones, se ha contratado a un experto para inspeccionar los vehículos durante su paso por un puesto de control. El período de contratación del experto tiene una duración de $D$ días consecutivos, con fecha de inicio a determinar. Adicionalmente, se cuenta con un listado de $n$ entradas, indicando los días $d_1, d_2, ..., d_n$ en que los camiones de la empresa pasarán por el puesto de control.

Se desea conocer la máxima cantidad $c$ de camiones que el experto puede llegar a inspeccionar durante su período de contratación, y un día $d$ para tomar como fecha de inicio de forma tal que efectivamente logre inspeccionar dicha cantidad de camiones.

Formalmente, dado $D$ un numero natural y una lista no vacía de $n$ números naturales $d_1, d_2, ..., d_n$ (no necesariamente distintos), se desea hallar un valor $d$ natural de forma tal que el intervalo $[d, d + D)$ contenga a la máxima cantidad posible de elementos de la lista. Además, se desea conocer la cantidad total $c$ de elementos contenidos en el intervalo.

Cada instancia del problema, así como su solución, se codifica como una lista de naturales separados por espacios, representando los siguientes valores:

\begin{tabular}{ll}
Entrada: &$D\;n\;d_1\;d_2\;...\;d_n$ \\
Salida:  &$d\;c$
\end{tabular}

\subsubsection{Ejemplos y observaciones}

La lista de días de llegada de los camiones no necesariamente se encuentra ordenada, y además puede contener días repetidos, que deben ser contados con su debida multiplicidad al computar la solución. Esto es razonable ya que muchos camiones pueden llegar al puesto de control en un mismo día. Por lo tanto, el siguiente es un ejemplo de instancia válida y su respectiva solución:

\begin{tabular}{ll}
Entrada: &$1\;5\;23\;23\;23\;23\;2$            \\
Salida:  &$23\;4$ \hspace{10mm}
\end{tabular}

Por otro lado, es importante notar que si bien la cantidad máxima $c$ de camiones inspeccionados es única, la fecha de inicio del período de contratación no lo es. Distintos valores para $d$ pueden generar intervalos $[d, d + D)$ incluyendo igual cantidad de elementos de la lista.

\begin{tabular}{ll}
Entrada: &$3\;2\;6\;7$            \\
Salida:  &$5\;2$ \hspace{10mm} ó  \\
Salida:  &$6\;2$ \hspace{10mm}
\end{tabular}