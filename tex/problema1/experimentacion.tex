La primer etapa de experimentación consistió en verificar empíricamente la cota de complejidad temporal obtenida teóricamente para el algoritmo completo.

Dado que la complejidad temporal de la solución es dominada por la etapa de ordenamiento, y que el algoritmo utilizado para este fin pertenece a librerías estándares, la experimentación subsiguiente se realizó sobre instancias donde la lista de entrada se encuentra ordenada, eliminando la instrucción de ordenamiento. Esto permitió constatar si el ciclo final incurre efectivamente en un costo a lo sumo lineal, y verificar la preponderancia del ordenamiento como parte de la solución.